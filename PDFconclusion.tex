\sloppy
\setlength{\parindent}{0em}

\section{О новых командах doxygen}
\textbf{@fn\sk}\newline
\textit{Данная команда позволяет doxygen понять, что данный блок комментариев принадлежит именно функции. Эта команда нужна, когда мы, к примеру, решили написать описание функции(почему-то) не рядом с ней, а через блок комментариев. Также это уместно, когда мы невероятным образом решили генерировать названия функций с помощью дефайна(например, чтобы автоматически генерировались названия вида)}

\begin{center}
НАЗВАНИЕ\_ФУНКЙИИ + ТИП\_ДАННЫХ
\end{center}

\textit{тогда названия функций в коде будут выглядеть примерно так:}

\begin{center}
\textbf{void DEFINE(vector\_function)(const VECTOR\_TYPE *elem)}
\end{center}

\textit{Вот в таких ситуациях, к примеру, можно использовать @fn. Похожие функции существуют и для других объектов, например: @enum, @class}\\[3mm]

\textbf{@addtogroup [group name]}\newline
\textit{Создает(если такая группа еще не создана) новую группу с именем 'group name' и добавляет в нее информацию о функции, структуре или другом объекте. В документации html, например, потом описания функций будут объединены в специальные группы(то есть вынесены на отдельные страницу)}\\[3mm]

\textbf{@a или @e или @em}\newline
\textit{Выделяет одно слово и прописывает его курсивом. \textcolor{red}{Только с @em можно использовать выделение многих слов: <em>много слов</em>} }\\[3mm]

\textbf{@arg @li}\newline
\textit{Эквивалент begin\{itemize\} в латехе. @li - эквивалент}\\[3mm]

\textbf{@b}\newline
\textit{Жирный шрифт, притом эту команду можно использовать как <b> несколько слов </b>, однако аналогов у нее нет}\\[3mm]

\textbf{@c}\newline
\textit{Выделяет одно слово и делает его шрифта consolas или подобным шрифтом. Аналоги - @p, @tt, но только @tt, @c можно испльзовать с множеством слов, @p почему-то вообще не работает}\\[3mm]

\textbf{(1)@copydoc [name] (2)@copybrief [name] (3)@copydetails [name]}\newline
\textit{(1) - копирует документацию объекта полностью\\(2) - копирует лишь быстрое описание объекта (то есть только то, что написано как @brief ...)\\ (3) - копирует дополнительную информацию об объекте (например аргументы)}\\[3mm]

\textbf{@mainpage [title] ...}\newline
\textit{Если написать эту команду с текстом после нее, то можно получить текст титульной страницы всего проекта. Хорошо использовать такую комаду вместе с @subpage [identifier]}\\[3mm]

\textbf{@subpage [title] - @page [title] ...}\newline
\textit{@subpage [identifier] позволяет сделать ссылку на страницу "@page [identifier]", текст после "@page [identifier]" добавит описание для данной страницы}\\[3mm]

\textbf{@emoji :name:}\newline
\textit{добавляет стикер из стандартного перечня стикеров в гитхабе(можно посмотреть по сслыке: https://gist.github.com/rxaviers/7360908.)}\\[3mm]

\textbf{@code[.extencion]}\newline
\textit{добавляет блок кода в документацию. Например, нужно добавить пример использования библиотеки на @mainpage или @subpage}\\[3mm]

\textbf{@verbatim - @endverbatim}\newline
\textit{Использоуется, когда нужно сделать текст похожиим на шрифт consolas}\\[3mm]

Текст, который пишется внутри @\{ ... @\} при дополнительной табуляции идет в блок \textrm{"}Detailed Description\textrm{"} - например при использовании @addtogroup 